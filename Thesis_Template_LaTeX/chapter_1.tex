\chapter{Introduction}\label{Chapter_1}
\section{Section-1}
\lipsum[1-5] %to generate filler text 
\subsection{Subsection-1}
\lipsum[1-2]
\begin{table}[ht]
  \centering
 \renewcommand{\arraystretch}{2}
  \begin{tabular}{|c|c|c|}
    \hline
    \textbf{Column 1} & \textbf{Column 2} & \textbf{Column 3} \\
    \hline
    Row 1, Cell 1 & Row 1, Cell 2 & Row 1, Cell 3 \\
    \hline
    Row 2, Cell 1 & Row 2, Cell 2 & Row 2, Cell 3 \\
    \hline
    Row 3, Cell 1 & Row 3, Cell 2 & Row 3, Cell 3 \\
    \hline
    Row 4, Cell 1 & Row 4, Cell 2 & Row 4, Cell 3 \\
    \hline
  \end{tabular}
  \caption{A Simple Table with 4 Rows and 3 Columns}
  \label{tab:simple-table} 
\end{table}

Here is an example of citing a paper \cite{sweinberg1989}.
\lipsum[1-3] %to generate filler text